\documentclass{standalone}
\usepackage{tikz}
\usetikzlibrary{shapes.geometric, positioning, calc}

\begin{document}

\begin{tikzpicture}[
    scale=0.7,
    octagon/.style={
        regular polygon,
        regular polygon sides=8,
        minimum size=4cm,
        draw=black,
        thick,
        fill=white
    },
    vertex_A/.style={fill=red!30, circle, inner sep=2pt},
    vertex_B/.style={fill=blue!30, circle, inner sep=2pt},
    arrow_style/.style={
        ->,
        thick,
        red!80!black,
        dashed,
        line width=1.2pt
    },
    label_style/.style={font=\footnotesize, inner sep=1pt}
]

% ===== 调整后的布局 (增大间距) =====
\node[octagon, rotate=-45] (P0) at (0,0) {}; % 中央八边形

% 右上角八边形
\node[octagon, rotate=90] (P1) at (7,10) {}; % 增加X/Y方向距离

% 左上角八边形
\node[octagon, rotate=180] (P2) at (-7,10) {}; % 增加X/Y方向距离

% ===== 顶点标记 =====
% P0
\node[vertex_A] at (P0.corner 1) {};
\node[vertex_A] at (P0.corner 2) {};
\node[vertex_B] at (P0.corner 3) {};
\node[vertex_B] at (P0.corner 4) {};
\node[label_style] at ($(P0.corner 1)+(0.5, 0.0)$) {$A_0$};
\node[label_style] at ($(P0.corner 3)+(0.0, 0.5)$) {$B_0$};

% P1 (旋转90度后的顶点)
\node[vertex_A] at (P1.corner 1) {};
\node[vertex_A] at (P1.corner 2) {};
\node[vertex_B] at (P1.corner 3) {};
\node[vertex_B] at (P1.corner 4) {};
\node[label_style] at ($(P1.corner 1)+(-0.5,0.0)$) {$A_1$}; % 调整标签位置
\node[label_style] at ($(P1.corner 3)+(-0.5,-0.5)$) {$B_1$}; % 调整标签位置

% P2 (旋转180度后的顶点)
\node[vertex_A] at (P2.corner 1) {};
\node[vertex_A] at (P2.corner 2) {};
\node[vertex_B] at (P2.corner 3) {};
\node[vertex_B] at (P2.corner 4) {};
\node[label_style] at ($(P2.corner 1)+(-0.5,-0.5)$) {$A_2$}; % 调整标签位置
\node[label_style] at ($(P2.corner 3)+(0.5,-0.5)$) {$B_2$}; % 调整标签位置

% ===== 边标签 =====
% P0 (中心)
\node[label_style] at ($(P0.corner 1)!0.5!(P0.corner 2)+(-0.5,-0.2)$) {$e_1^0$};
\node[label_style] at ($(P0.corner 3)!0.5!(P0.corner 4)+(0.5,-0.2)$) {$e_3^0$};

% P1 (左下,旋转90度)
\node[label_style] at ($(P1.corner 1)!0.5!(P1.corner 2)+(0.5,0.0)$) {$e_1^1$};
\node[label_style] at ($(P1.corner 3)!0.5!(P1.corner 4)+(0.0,0.5)$) {$e_3^1$};

% P2 (左上,旋转180度)
\node[label_style] at ($(P2.corner 1)!0.5!(P2.corner 2)+(0.0,0.5)$) {$e_1^2$};
\node[label_style] at ($(P2.corner 3)!0.5!(P2.corner 4)+(-0.5,0.0)$) {$e_3^2$};

% ===== 箭头路径 (调整后的路径) =====
% 箭头1: e1^0 -> e3^1 (P0到P1)
\draw[arrow_style] ($(P0.corner 1)!0.5!(P0.corner 2)$) to[out=0, in=270]
    node[midway, above left, label_style] {$e_1^0 \leftrightarrow (e_3^1)^{-1}$}
    ($(P1.corner 4)!0.5!(P1.corner 3)$);

% 箭头2: e1^1 -> e3^2 (P1到P2)
\draw[arrow_style] ($(P1.corner 1)!0.5!(P1.corner 2)$) to[out=180, in=0]
    node[midway, below left, label_style] {$e_1^1 \leftrightarrow (e_3^2)^{-1}$}
    ($(P2.corner 3)!0.5!(P2.corner 4)$);

% 箭头3: e1^2 -> e3^0 (P2到P0)
\draw[arrow_style] ($(P2.corner 1)!0.5!(P2.corner 2)$) to[out=270, in=180]
    node[midway, above right, label_style] {$e_1^2 \leftrightarrow (e_3^0)^{-1}$}
    ($(P0.corner 4)!0.5!(P0.corner 3)$);

\end{tikzpicture}
\end{document}